\documentclass[a4paper,12pt,titlepage]{article}

\title{Customize}
\author{\href{smhhaoo@126.com}{smher}}

\usepackage[xetex,colorlinks]{hyperref}
\usepackage{calc}
\usepackage[UTF8]{ctex}
%\usepackage{ams-mdbch}

%\addtolength{\hoffset}{-0.5cm}
%\addtolength{\textwidth}{1cm}

\begin{document}
\maketitle
\tableofcontents

\newpage
\section{\textit{你好}}
你好, \LaTeX !!!\\
{\par \Huge \textit{来自中文的问候!!} }
\section{new cmd \& env \& package}
\subsection{new cmd}
Use \verb|\newcommand{text}[num]{def}| to define own command.\\
For example:\\
\newcommand{\owncmd}[1]{This is the \emph{#1} Short 
					Introduce to \TeX}
\begin{itemize}
\item \owncmd{not so}
\item \owncmd{very}
\end{itemize}

Other command:\\
\verb|\renewcommand{text}[num]{def}| \emph{Or} \verb|\providecommand|

\subsection{new env}
Use \verb|\newenvironment{text}[args]{begdef}{enddef}| to define a new environment.
For example: \\
\newenvironment{king}{\rule{1ex}{1ex} %
						\hspace{\stretch{1}}}{\hspace{\stretch{1}} %
						\rule{1ex}{1ex}}

\begin{king}
my humble subjects \ldots
\end{king}

\subsection{extra space}
\newenvironment{simple}{\noindent}{\par\noindent}
\begin{simple}
See the space \\ to the left.
\end{simple}
Same here.

\newenvironment{correct}{\noindent\ignorespaces}{\par\noindent\ignorespacesafterend}

\begin{correct}
No space \\ to the left.
\end{correct}
Same here.

\subsection{cmd \LaTeX}
Add \verb|\usepackage{ifthen}|, to the start of \TeX file.\\
Use \verb|latex '\newcommand{cmd}{def}\input{file}'| to use the new command.\\

\subsection{Create own package}
See the manul for more details.
Include \verb|\ProvidesPackage{name}[release]| \& \verb|\newcommand{text}{def}| \&
\verb|\newenvironment{name}{begdef}{enddef}| and so on \ldots \\

\noindent\emph{Note:} \\
the file type is \emph{.sty} !

\section{Fonts}
\subsection{Change Fonts}
See follow example :\\
{\small The small and \textbf{Bold}} Romans ruled
{\Large all of great big \textit{Italy}. }\\
T Here ?\\
He likes {\LARGE large and {\small small} letters}.\\
\textsc{PostScripe}\\
\verb|\par| is a space line \\

\begin{itemize}
\item \verb|\textrm{text}| -- \textrm{text}
\item \verb|\texttt{text}| -- \texttt{text}
\item \verb|\textmd{text}| -- \textmd{text}
\item \verb|\textup{text}| -- \textup{text}
\item \verb|\textsl{text}| -- \textsl{text}
\item \verb|\textsf{text}| -- \textsf{text}
\item \verb|\textbf{text}| -- \textbf{text}
\item \verb|\textit{text}| -- \textit{text}
\item \verb|\textsc{text}| -- \textsc{Hello}
\item \verb|\textnormal{...}| -- \textnormal{Document font}
\end{itemize}

{\Large Don't read this !
It is not ture. You can believe me. \par
}

Hello, new line...

{\Large Don't read this !
It is not ture. You can believe me.
} \par

Hello, new line ...

\noindent\verb|\tiny| --- {\tiny hello} \\
\verb|\Huge|---{\Huge hello} \\
\verb|\huge|---{\huge hello} \\
\verb|\large|---{\large hello} \\
\verb|\normalsize|---{\normalsize hello} \\

\subsection{use newcommand}
\newcommand{\oops}[1]{\textbf{#1}}

Do not \oops{enter} this room, it's occupied by \oops{machines} of unknown origin and purpose.

\underline{\textbf{hello !}}

\section{Space}
Use \verb|\linespread{factor}| before \verb|\begin{document}| at the beginning of file. \\
Better to use \verb|\setlength{text}{length}|, for example :\\
\begin{quote}
{\setlength{\baselineskip}{1.5\baselineskip}
This paragraph is typeset with the baseline skip set to 1.5 of what it was before. Note the par command at the end of the paragraph. \par
}
This paragraph is normal with the baseline skip set to 1.5 of what it was before. Note the par command at the end of the paragraph.
\end{quote}

\subsection{paragraph layout}
There are two parameters to decide the layout of paragraph :\\
{\centering
\verb|\setlength{\parindent}{0pt}|\\
\verb|\setlength{\parskip}{1ex plus 0.5ex minus 0.2ex}| \\
}
Hello ...\\
\indent Hello ...\\

\subsection{horizontal space}
Use \verb|\hspace{length}| to increase horizontal distance.\\
{
This \hspace{1.5cm} is a space of 1.5 cm .
}

Use \verb|\stretch{text}| to fulfill the whole line.\\
x \hspace{\stretch{1}} x \hspace{\stretch{3}} x. \\

Use \verb|em|\\
{\Large{} \par Big \hspace{1em} y} \\
{\tiny{} tiny \hspace{1em} y} \\

Hello ...

\subsection{vertical space}
Use \verb|\vspace{text}| to increase vertical space ...\\
Some text \ldots

\vspace{\stretch{1}} 
this goes onto the last line of the page. \pagebreak

Hello ...

\section{page layout}

\section{more length}
\flushleft
\newenvironment{vardesc}[1] %
					{\settowidth{\parindent}{#1:\ } %
					\makebox[0pt][r]{#1:\ }}{}

\begin{displaymath}
a^2 + b^2 = c^2
\end{displaymath}
\begin{vardesc}{Where}$a$,
$b$ -- are adjoin to the right angle of a right-angled triangle.

$c$ -- is the hypotenuse of the triangle and feels lonely.

$d$ -- finally does not show up here at all. Isn't that puzzling?
\end{vardesc}

\section{Box}
Use \verb|\parbox[pos]{text}{text}| to put a paragraph into a box!\\
\emph{\large Or} \\
\verb|\begin{minipage}[pos]{width}| text \verb|\end{minipage}|  \\

\makebox[\textwidth]{central}\par
\makebox[\textwidth][s]{spread}\par
\framebox[1.1\width]{Guess I'm framed now!}\par
\framebox[0.8\width][r]{Bummer, I am too wide}\par
\framebox[1cm][l]{Never mind, so am I}
Can you read this...\\

Hello ...\\
% em mean horizontal distance; ex mean vertical distance.
\raisebox{0pt}[0pt][0pt]{\Large %
\textbf{Aaaa\raisebox{-0.3ex}{a} %
\raisebox{-0.7ex}{aa}
\raisebox{-1.2ex}{r}
\raisebox{-2.2ex}{g}
\raisebox{-4.5ex}{h}}
}
He should but not even the next one in line noticed that something terrible had happened to him.

\section{Dimension}
\rule{3mm}{.1pt} %
\rule[-1mm]{5mm}{1cm}
\rule[1mm]{1cm}{3mm}

\subsection{Struts}
\begin{tabular}{|c|}
\hline
\rule{1pt}{4ex}Pitprop \ldots\\
\hline
\rule{0pt}{4ex} Strut\\
\hline
\end{tabular}

\begin{flushright}
{
\Huge
\textit{END.} \\
~\\  % add new sparse line !!!
\today
}
\end{flushright}

\end{document}