\chapter{Saliency Detection}
This chapter includes papers about saliency detection.
\section{Frequency-tuned Salient Region Detection}
\subsection{Abstract}
In this paper, we introduce a method for salient region detection that outputs full resolution saliency maps with well-defined boundaries of salient
objects. These boundaries are preserved by retaining substantially more frequency content from the original image than other existing techniques. Our method exploits fea-
tures of color and luminance, is simple to implement, and is computationally efficient. We compare our algorithm to five
state-of-the-art salient region detection methods with a frequency domain analysis, ground truth, and a salient object segmentation application. Our method outperforms the five algorithms both on the ground-truth evaluation and on the segmentation task by achieving both higher precision and better recall.
\subsection{Contents}
\subsubsection{Related work}
Saliency estimation methods can broadly be classified as:
\begin{itemize}
\item biologically based
\item purely computational 
\item combination of above two approaches
\end{itemize}

Itti base their method on the biologically plausible architecture proposed by Koch and Ullman. They determine center-surround contrast using a \textbf{Difference of Gaussians}
(DoG)\index{DoG}. Frintrop presetn a method inspired by Itti's method, but they compute \textbf{center-surround differences} with square filters and use integral images to speed up the calculations.

Other methods are purely computational 
and are not based on biological vision principles. Ma and
Zhang  and Achanta et al. estimate saliency us-
ing center-surround feature distances. Hu et al.  es-
timate saliency by applying heuristic measures on initial
saliency measures obtained by histogram thresholding of
feature maps. Gao and Vasconcelos maximize the mu-
tual information between the feature distributions of center
and surround regions in an image, while Hou and Zhang
 rely on frequency domain processing.
 
The third category of methods are those that incorporate
ideas that are partly based on biological models and partly
on computational ones. For instance, Harel et al. create
feature maps using Itti’s method but perform their normal-
ization using a graph based approach. Other methods use
a computational approach like maximization of information
 that represents a biologically plausible model of saliency
detection.
\subsubsection*{Limitations}
The saliency maps generated by most methods have low resolutioni. Itti's method produces saliency maps that are just $1/{256^{th}}$

\subsubsection*{Frequency-tuned Saliency Detection}
\textbf{DoG}

DoG : Difference of Gaussians.  DoG filter is widely used in edge detecion since it closely and efficiently approximates the Laplacian of Gaussian (LoG) filter, cited as the most satisfactory operator for detecting intensity changes when the standard deviations of the Gaussians are in the ratio 1 : 1.6. The DoG has also been used for interest point detection and  saliency detection.  The DoG filter is given by : 
\begin{equation}
\begin{gathered}
DoG(x, y) = \frac{1}{2\pi}\left[
\frac{1}{\delta_1^2}e^{-\frac{x^2 + y^2}{2\delta_1^2}} - 
\frac{1}{\delta_2^2}e^{-\frac{x^2 + y^2}{2\delta_2^2}}
\right]\\
= G(x, y, \delta_1) - G(x, y, \delta_2)
\end{gathered}
\end{equation}
where $\delta_1$ and $\delta_2$ are the standard deviatioins of the Gaussian $(\delta_1 > \delta_2)$.

A DoG filter is a simple band-pass filter whose passband
width is controlled by the ratio $\delta_1 : \delta_2$. 
\subsection{Conclusion}
